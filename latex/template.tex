\documentclass[a4paper,titlepage,12pt]{article}
\usepackage[utf8]{inputenc}
\usepackage[T1]{fontenc}
\usepackage[margin=1in]{geometry}
\usepackage[british]{babel}
\usepackage{csquotes}
\usepackage[backend=biber,style=apa]{biblatex}
\usepackage{parskip}
\usepackage{graphicx}
\usepackage{hyperref}
\usepackage{listings}
\usepackage{multirow}
\usepackage[usenames,dvipsnames]{color}
\usepackage{makeidx}
\makeindex
\DeclareLanguageMapping{british}{british-apa}
\addbibresource{template.bib}
\hypersetup{
	colorlinks,
	pdfauthor=Delan Azabani,
	pdftitle=Untitled document
}
\lstset{basicstyle=\ttfamily, basewidth=0.5em}

\title{Untitled document}
\date{September 23, 2012}
\author{Delan Azabani}

\pagenumbering{gobble}
\thispdfpagelabel{0}

\begin{document}

\maketitle
\pagenumbering{roman}

\vspace*{0.2\vsize}
\hspace{0.1\hsize}
\begin{minipage}{0.8\hsize}

\section*{Abstract}

The contents of this document serve as a placeholder, demonstrating some of
the features of {\LaTeX} and \textsc{Bib\TeX} such as a table of contents, a
list of figures, a list of tables, citations, a list of references, as well as
an index.

\end{minipage}

\newpage
\tableofcontents
\newpage
\renewcommand{\listfigurename}{List of figures}
\listoffigures
\newpage
\renewcommand{\listtablename}{List of tables}
\listoftables
\newpage
\pagenumbering{arabic}

\section{The channel capacity of a noiseless, stateful channel}

\textcite{Shannon1948} established the field of \emph{information
theory}\index{information theory}, defining a general model of devices
communicating over discrete and continuous channels, where these channels are
optionally subject to noise. Concepts such as information \emph{entropy}
\index{entropy} and \emph{redundancy}\index{redundancy} are also introduced in
Shannon's work (\cite{Shannon1948}).

Of particular relevance, Shannon suggests a means of calculating the channel
capacity of a discrete, noiseless, and stateful protocol, in which a subset of
symbols may be allowed for each state transition, and each symbol may have a
unique but otherwise fixed duration. Because the only information about each
state that is used in this calculation is its duration, any given symbol may
assume different durations where it is valid for a given state transition.
One may simply think of them as different symbols, and more broadly, think of
symbols as unique to a given state transition, regardless of their appearance.

We can form an algebraic representation of the durations of the valid symbols
that may be used to transition from and to each given pair of states, arrange
these expressions in a square matrix, find its determinant, and then find the
logarithm of the maximum real zero of the resultant expression. This
expression may be represented mathematically as

\begin{equation}
	\left|\sum_{s}{W^{-b_{ijs}}}-\delta_{ij}\right|=0
\end{equation}

where $W$ is an arbitrary variable that is used in the evaluation of the zeros
of the function, $i$ and $j$ identify the source and destination states, as
well as identify the rows and columns of the matrix, $s$ identifies a given
symbol that is valid for a given state transition, $b$ is the duration of a
given symbol, and $\delta$ is a constant that is equal to one where $i = j$,
or zero where a given state transition changes the current state.

\newpage
\begin{sloppypar}
	\printbibliography
\end{sloppypar}
\newpage
\printindex

\end{document}
